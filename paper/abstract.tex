\begin{abstract}
Personal information in passwords was understudied due to various reasons. To fill this gap, in this paper we dissect user passwords from a leaked dataset to study how and to what extent do user personal information resides in their passwords. We present most popular password structures as expressed by personal information and show high correlation between passwords and personal information. We also found that male and female behave differently regarding to generating passwords with personal information. Then a quantification metric -- Coverage -- that describes the correlation between passwords and personal information is carried out in this work. Seeing the potential of cracking passwords on top of our analysis, We succeeded developing a semantics-richer Probabilistic Context-Free Grammars method called Personal-PCFG. Personal-PCFG cracks passwords much faster than the state-of-art technique and makes online attacks much easier and more feasible. To defend such semantics-aware attacks, we propose to use distortion functions that are chosen by users to mitigate unwanted correlation between personal information and passwords.

\end{abstract}

% A category with the (minimum) three required fields
\category{}{Security and privacy}{}[Human and societal aspects of security and privacy]
%A category including the fourth, optional field follows...
\category{}{General and reference}{}[Metrics]

\terms{Security}

\keywords{passwords, password cracking, data processing, password protection}