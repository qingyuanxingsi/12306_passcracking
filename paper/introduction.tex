
\section{Introduction}
Text-based passwords still remain a dominating and irreplaceable
authentication method in foreseeable future. Although people have
proposed different authentication mechanisms, no alternative can bring
all benefits of passwords without introducing any extra burden to
users \cite{bonneau2012quest}. However, passwords have long been
criticized to be one of the weakest link in authentication.  Due to
human-memorability requirement, user passwords are usually far from
true random strings
\cite{bonneau2012science,malone2012investigating,narayanan2005fast,veras2012visualizing,yan2004password}. For
example, ``secret'' is more likely a human-chosen password than
``zjorqpe''. In other words, human users are prone to choose weak
passwords simply because they are easy to remember.  As a result, most
passwords are chosen within only a small portion of the entire large
password space, being vulnerable to brute-force and dictionary attacks.


%help both users and system administrators better enhance passwords strength. a very common way is to restrict password creation policies by requiring non-alphabets in passwords. The bad news is that these policies do not keep passwords from vulnerable because users tend to simply mangle their passwords to fulfill the restriction \cite{weir2010testing}, making the security improvement very limited. 

To increase password security, online authentication systems start to
enforce more strict password policies. Meanwhile, currently most
websites have password strength meters deployed to help users choose
secure passwords. However, these meters are proved to be ad-hoc and
inconsistent \cite{de2014very}. To better assess the strength of
passwords, we need to have a deeper understanding on how users
construct their passwords. If an attacker knows exactly how users
create their passwords, guessing their passwords will become much
easier. Meanwhile, if a user is aware of the potential vulnerability
induced by a commonly used password creation method, the user can
avoid using the same method for creating passwords.


Toward this end, researchers have done a lot of work to unveil the
structures of passwords. Traditional dictionary attacks on passwords
have shown that users tend to use simple dictionary words to construct
their passwords~\cite{hellman1980cryptanalytic,
  morris1979password}. Languages are also vital since users tend to
use their own languages when constructing
passwords~\cite{bonneau2012science}. Besides, passwords are mostly
phonetically memorable~\cite{narayanan2005fast} even they are not
simple dictionary words. It is also indicated that users may use
keyboard strings such as ``qwerty" and ``qweasdzxc", trivial strings
such as ``password", ``123456", and date strings such as ``19951225"
in their passwords~\cite{li2014large, schweitzer2009visualizing,
  veras2012visualizing}. However, most studies only discover
superficial password patterns, and the semantic-rich composition of
passwords is still mysterious to be fully uncovered. Fortunately, an
enlightening work investigates on how users generate their passwords
by learning the semantic patterns in
passwords~\cite{veras2014semantic}.

In this paper, we study password semantics from a different
perspective---personal information.  We utilize a leaked password
dataset, which contains personal information, from a Chinese website
for this study.  We first measure the usage of personal information in
password creation and present interesting observations. We are able to
obtain the most popular password structures with personal information
embedded. We also observe that male and female behave differently when
using personal information in password creation. Next, we introduce a
new metric called Coverage to accurately quantify the correlation
between personal information and user password. Since it considers
both the length and continuation of personal information in a
password, Coverage is a useful metric to measure the strength of a
password. Our quantification results using the Coverage metric confirm
with our direct measurement results on the dataset, showing the
efficacy of Coverage.  Moreover, Coverage is easy to apply and to be integrated with existing tools, such as password strength
meters to uncover the unnoticed corner of personal information.

To demonstrate the security vulnerability induced by using personal
information in passwords, we propose a semantics-rich Probabilistic Context-Free
Grammars (PCFG) method called Personal-PCFG, which extends PCFG~\cite{weir2009password} by considering personal
information symbols in password structures. Personal-PCFG is able to
crack passwords much faster than PCFG. It also makes an online attack
more feasible by drastically increasing guessing success rate. Finally, we present simple distortion
functions to defend against these semantics-aware attacks such as
Personal-PCFG or~\cite{veras2014semantic}. Our evaluation results
demonstrate that distortion functions can effectively protect
passwords by significantly reducing the unwanted correlation between
personal information and passwords.

Our study is based on a dataset collected from a Chinese
website. Although measurement results could be different in some
aspects if other datasets are used, our observations still show a
valid grand picture on how personal information is used in passwords.
As long as memorability plays an important role in password creation,
the correlation between personal information and user password
remains, regardless of which language users speak. We believe that our
work on personal information quantification, password cracking, and
password protection could be applicable for any other text-based
password datasets from different websites.

The remainder of this paper is organized as
follows. Section~\ref{personalinfo} measures how personal information
resides in user passwords and shows the gender difference in password
creation, based on a leaked dataset from an online ticket reservation
website. Section~\ref{correlationquantification} introduces the new
metric, Coverage, to accurately quantify the correlation between
personal information and user password.  Section~\ref{personalpcfg}
details Personal-PCFG and shows cracking results compared with the
original PCFG.  Section~\ref{passwordprotection} presents our
defense against semantics-aware attacks by applying simple distortion
functions on password creation. Section~\ref{sec:related} surveys
related work, and finally Section~\ref{sec:conclusion} concludes this
paper.
